\section{Model Assumptions}

%Controlling the spread of infections is by no means a new problem, and many models have previously been used to describe the distribution and spread of epidemic diseases.
%
%The models previous scientists used can mainly be divided into three categories. Brief overviews of the three by the sequence of their advent are as follows.
%\begin{enumerate}
%  \item \textbf{\emph{Standard SIR(susceptible-infected-removed) models and its deriveratives:}}The model is first put forward by \textbf{Kemraek} and \textbf{MeKendriek} in 1927\cite{SIR}and is the most classic model. Many successive models are based on it. It regards the epidemic area as a whole and divides people into different groups, such as susceptible group, infected group and removed group. It depends on differential equations to describe the relationship between volumes of different groups, thus making scientific and quantitative description. Details of the model will be explained in the successive sections.
%  \par Many other models are derived from this very basic SIR model, such as SIS model\cite{SIS} which deals with situations where a infected and later recovered individual can turn back to be a susceptible individual, and SEIR model\cite{SEIR} which deals with the situations where incubation period of the epidemic disease is not negligible. Previous scientists gained great achievements by using SIR model - they simulated the trends of epidemic diseases, calculated the rough volume of each group in different periods of epidemics and, most importantly, successfully analyse the contribution of different parameters to stability of system. 
%  \par Despite the fundamental and important standing of SIR model, its drawbacks are easy to realize. It is difficult to study geographic characters of the spread of diseases, because the model doesn't contain any geographic information. Also, each person in the system is not individually recognized and the complex relationships among individuals are hardly considered, since it regards the epidemic area as a homogeneous system - in another word, each person in the system is just like a molecule in a cup of water and the relationships of any two pairs are identical. Additionally, it is a deterministic model witch hardly resemble the factual system in which many fortuitous factors matter.
%  \item \textbf{\emph{Statistical model:}}This model is less intensely studied than the other two models. In this model, researchers mainly use possible functions whose forms are known but parameters unknown to fit the currently available data, aiming at predicting the trend of epidemics. Although the model is somewhat phenomenological and catch little intrinsic logic, it is quite useful in real practice and it allows us to have a quick view of the trend without getting convoluted into the depth of the problem. Anyhow, we will not go further into this kind of models.
%  \item \textbf{\emph{Spatial simulation model:}}This category of models are the focus recently. They mainly contains models using cellular automaton (CA), models considering networks among people(eg. small world network), models using GIS\cite{GIS}, etc. Take models using CA as an example. We define different types of points, which are put in lattices of a plane, and regard the spread of diseases between people as the interaction between nearby points. The process of evolution and spatial distribution of virus can be simulated with the aid of computer program. Clearly, due to the consideration of individuals separately, the results are more accurate than previous two categories.
%  
%  This category of models taking the relationship among people and geographic information into account has advantages over others when figuring out the geographic characters of the spread. Moreover, it can give more accurate results since its high resemblance of real situation. Accurate as the models are, they are highly complicated and some of them are difficult to apply both for the complexity of programming and great amount of computation.
%\end{enumerate}

\begin{itemize}
  \item \textbf{Population}\\In general, a large population means high potential of disease's spread. People in a populous area have a greater frequency of contacting the others than those in a sparsely populated area. According to the route of transmission, it is obvious that the probability of getting infected would be larger.
   
  \item \textbf{Traffic}\\Convinient traffic encourages population mobility, which contributes to the spread of EVD. However, It also encourages freightage, including medicine.
  
  \item \textbf{Medical Level}\\A society will be less affected by EVD if proper measures are taken efficiently and promptly.
   These measures include segregating patients and strengthening the sanitary control of public places. The manufacture of drugs and vaccines aiming at EVD is also an important part.
  
  \item \textbf{Regional Custom}\\Funeral is considered solemn in the African culture. The dead should be cleaned, kissed and touched before buried. This kind of culture facilitates EVD infections.
  
  \item \textbf{Other Social Factors}\\The spread of diseases is also influenced by factors like social development, health situation, individual's living condition, etc. These factors are not considered to simplify our models.
\end{itemize}