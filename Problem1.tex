\section{Problem 1: Design for Decision Model}
%\subsection{Strengths}
\begin{enumerate}
\item \textbf{Our model is simple and easy to understand} \\
Our model is the simplest model we can conceive to reflect the impact of concerned independent variables (factors regarding medication) and to solve the problem lifted by the question. 

Our single-city model is based on the most elegant model in the field of epidemiology - the SIR model, and we reconstruct the model (mainly add two clusters of people) in order to introduce concerned independent variable into our system. 

Our multi-city model is based on our single-city model and introduce only one `people flow` to obtain the geographic characteristic of the spread of disease.

\item \textbf{Our model is effective and in good agreement with the reality} \\
Simple as they are, they are effective in reflecting the complex relationships between numerous variables and parameters, and they not only reveal the intrinsic characteristics of the spread of disease itself but also successfully link factors of medication to the spread of disease.

Comparing with the data we have find from several resources, the results of our model not only correspond the general trend of the records but also resemble the reality in some critical features.

\item \textbf{Good extensibility} \\
Flow of people is a critical factor determining the spread of disease. Although our multi-city model only set the volume of people flow as a function of mere geographic distribution and population of cities, the determinants of people flow can be adjusted when other possible factors are considered. Then, the adjusted model can be applied to study the impact of other possible factors relating to epidemiology.

\end{enumerate}



%\subsection{Weaknesses}
\begin{enumerate}
\item \textbf{Our model is just a rough model} \\
For simplicity, we have neglected many potential parameters, variables or processes, and have made numerous assumptions. Eg. we did not consider the relationship between separate individuals and we did not dig deeper into the properties of social network which is a quite essential part determining the spread of disease. Some important general or specific factors are also neglected by us, a interesting example of which is a folk custom prevalent in the studied region that relatives kiss the death, which plays a significant role in the spread of disease and is categorized into \emph{Super Spread Event}(SSE) academically. 

\item \textbf{Our model is only a continuous model} \\
Numbers of people, number of shares of drug/vaccine, etc. are important quantities in all the process of modeling and computation. For simplicity, we regard the numbers as directly real numbers instead of integers. It is justifiable when the numbers are large, since the decimal part of the number is negligible; when the system scales down, however, the statistics dose not work and the outcome deviates a lot from reality.
\end{enumerate}