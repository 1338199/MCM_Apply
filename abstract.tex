\begin{abstract}
\par Airport taxi scheduling is a classic and practical problem. Reasonable decision-making scheme is conducive to maximize the driver's profit, and also help the airport to allocate driving resources efficiently. In this paper, the decision-making mechanism of whether the driver leaving or staying at the airport is designed, and the scheme of airport's passenger-taxi boarding is planned.
\par For questions 1 and 2, considering the driver aims to make more profits, we take the taxi's net income in a certain time interval as the decision standard. we take the particular time period from the taxi entering the "storage pool" to the completion of the passenger delivery. In this way, various factors can be comprehensively analyzed. By cluster analysis of timestamp, taxi density, flight density, length of stay, we can predict the queueing time of taxi at a certain timestamp. Factors can be quantified by airport data, approximation, probability statistics and other methods. After substituting the model into the one-day data of Pudong airport, we come to the conclusion that the net income of taxi waiting in line during 22:00-24:00 and rush hour will be higher than that of returning to the city, and the decision-making model has the greatest dependence on flight density.
\par For the third problem, the setting of boarding points should consider the number and the interval of boarding points. Therefore, we set up a dynamic solution, which uses genetic algorithm. And in order to solve the problem of taxi and passenger arrangement, A queuing service model about how many taxis correspond to one boarding point and how many passengers correspond to one boarding point is setting up. After substitute the data into the model we get the setting schemes: before six o'clock in the evening, the pedestrian volume is small, three boarding points are set on one side and two parking points are set on the other side, the time interval of each parking point is 22m, three taxis are arranged for each batch to go to one boarding point. After 6 p.m., there is a large pedestrian volume. So there will be 28 boarding points on both sides of the road. Each boarding point is 122 meters apart, and for each batch two taxis are arranged to go to same boarding point.
\par For the fourth question, our priority scheme is to add a short distance Lane in the queuing area for the latest taxis after delivering short distance passengers. As to whether the scheme can make the taxis' revenue more balanced, we build a model to verify. In the verification, we find that when there is no priority, the short-distance vehicle revenue is indeed inferior to that of the long-distance vehicle. And the average revenue of the short-distance vehicle can be improved with our solution.
\end{abstract}
