\section{Model Analysis}
\subsection{Analysis of task 1}

\par First, we have to determine the driver's decision criteria. Here it's the net income of the queuing car and the returning car in the time interval of the queuing taxi from start queuing to finish passengers delivering , so that important factors such as queuing time, airport passenger source, no-load cost can be taken into account. Here, the net income is calculated by subtracting the total fuel cost from the total fare income. For queuing vehicles, the net income is related to mileage, single ride fare and oil price. For return vehicles, the net income is also related to the saved queuing time, vehicle speed, probability of carrying urban passenger, weather, road conditions and other factors. Vehicle speed and mileage can be obtained by approximation; pricing rules and oil prices can be inquired; urban passenger carrying probability can be calculated by statistics; weather influence are related to visibility; road conditions can be estimated by the traffic flow obeying Poisson distribution; queuing time cannot be obtained directly, which is affected by timestamps, number of passengers and number of taxis, while drivers cannot determine the number of passengers directly, and what can be determined is the number of flights arrived in this period. Therefore, cluster analysis is carried out for (time point, flight density, taxi density, vehicle dwell time) to predict the taxi queuing time in the new situation.

%\subsubsection{Biological factors}
%\begin{itemize}
%  \item \textbf{Infectiousness }\\ Infectiousness directly affects the disease's infection rate, which is defined as morbidity rate for uninfected people but exposed to virus carriers. And infectiousness is mainly affected by virulence of the virus and individual's immunity. In most cases, diseases' infection rate is regarded as constant.\\ 
%      
%  \item \textbf{Lethality }\\ Lethality is how capable a disease is of causing death. It can be described by lethality rate, the ratio of the dead to the total patients. EVD has an average fatality rate of $55\%-80\%$. Moreover, Individuals who was infected and cured won't be infected again in next ten years.
%       
%  \item \textbf{Incubation}\\ Incubation means that there is a period during which the infected shows no sign or symptom of the disease and is not contagious\cite{CDC}. Therefore, there may be difficulty to segregate people with infectious diseases. EVD's incubation period is $2$ to $21$ days.
%      
%  \item \textbf{Route of Transmission} \\EVD is mainly spread through direct contact with blood and body fluids of infected ones\cite{CDC}. These routes of transmission are highly connected to public health situation and personal hygiene.
%      
%  \item \textbf{Environmental Condition}\\ Ebola virus has moderate tolerance to heat, which indicates that EVD maintains a high infectiousness in most of human settlements. However, Ebola virus will be inactivated when exposed to a temperature over $60^{\circ}$  for $60$ minutes. Environment also affects the spread of EVD in aspect of natural reservoir. Though the reservoir remains unknown, it is reasonable to infer that contact with wide animals adds to the probability of getting infected.
%      
%%  \item 
%
%\end{itemize}
%
%
%\subsubsection{Social factors}
%\begin{itemize}
%  \item \textbf{Population}\\In general, a large population means high potential of disease's spread. People in a populous area have a greater frequency of contacting the others than those in a sparsely populated area. According to the route of transmission, it is obvious that the probability of getting infected would be larger.
%   
%  \item \textbf{Traffic}\\Convinient traffic encourages population mobility, which contributes to the spread of EVD. However, It also encourages freightage, including medicine.
%  
%  \item \textbf{Medical Level}\\A society will be less affected by EVD if proper measures are taken efficiently and promptly.
%   These measures include segregating patients and strengthening the sanitary control of public places. The manufacture of drugs and vaccines aiming at EVD is also an important part.
%  
%  \item \textbf{Regional Custom}\\Funeral is considered solemn in the African culture. The dead should be cleaned, kissed and touched before buried. This kind of culture facilitates EVD infections.
%  
%  \item \textbf{Other Social Factors}\\The spread of diseases is also influenced by factors like social development, health situation, individual's living condition, etc. These factors are not considered to simplify our models.
%\end{itemize}

\subsection{Analysis of task 2}

\par First of all, we need to determine an airport, take the relevant samples from the taxi and flight data, and set some parameters. Then bring the samples (time point, flight density, taxi density, vehicle dwell time) into the cluster model, adjust the cluster number k, label variable distance weight and other relevant parameters, find the best cluster number, and then bring the new samples without queuing time into the decision-making model to obtain the corresponding decision. This paper gives the selection scheme of taxi drivers in different situations. 

As for the decision model's dependence on various factors. We combine the decision-making model with time period, flight density and taxi density. Observe and analyze their relations by drawing statistics.

Finally, tested by the actual situation, such as public transportation, urban roads and some official statistical data, the decision-making model is reasonable. Then, the sensitivity matrix is obtained by using sensitivity analysis on the decision-making model. And we can obtain the variables with the the model's highest dependence.

\subsection{Analysis of task 3}

\par For this question, we play the role of management department, need to set up boarding points, and arrange taxis and passengers to the corresponding boarding points, the goal is to make the total boarding efficiency the highest. There are only two lanes. In order to increase efficiency, boarding points are set on both sides of the road, and the number of passengers at each side of the boarding point should be as average as possible. It is also necessary to allocate the same number of taxis to the boarding point, as well as to find out how many taxis are dispatched to the same boarding point at a time.

In order to make the total boarding efficiency the highest, that is, the total waiting time of passengers is the least. The waiting time of passengers is divided into two parts, one is the time when the passengers arrive at the corresponding boarding point, the other is the time when the passengers wait in line at the boarding point until they become the first person of the line. When the passenger flow is large, and a certain flight arrives at the airport, the passengers of the previous flight are possibly still waiting for the taxis, so it is unnecessary to calculate when they arrive at the corresponding boarding point, because they are already in the waiting line. Therefore, the model needs to be classified into two types, which are respectively used in the scenarios with large and small passenger flow. Moreover, when the passenger flow is large, in order to meet the needs of passengers, the number of boarding points will inevitably increase, but not as much as possible, because there will be "ghost traffic jam" phenomenon affecting the average speed of taxis, so we use the logarithmic model of speed and traffic density to simulate this phenomenon when the traffic flow is large, and the setting of boarding points needs to consider the interval to ensure safety and convenience for passengers. Finally, we establish the relationship between the number of boarding points, the interval, the scheduling scheme and the total waiting time of passengers. We use genetic algorithm to find the number of boarding points, the interval and the scheduling scheme with the minimum waiting time of passengers, and use the data of Pudong Airport as an example to analyze.

\subsection{Analysis of task 4}

\par The purpose of establishing the short-distance vehicle priority scheme is to reduce the queuing time of short-distance vehicles. The effective method is to set up the short-distance vehicle priority channel, other vehicles can't enter the queue until the short-distance vehicle reaches the passengers. 

The equilibrium of revenue is reflected in the variance of revenue of all taxis. Therefore, the average revenue model of taxis is established. It is assumed that the short-distance vehicle will return to the airport every time until it reaches the long-distance passenger, and the long-distance vehicle will not return to the airport after completing the passenger journey. The appropriate time period for the short-distance vehicle is from start queuing to finish delivering the long-distance passenger, and the appropriate time interval for the long-distance vehicle is from start queuing to completion of passenger delivering. According to this, we can calculate the hourly income of short-distance vehicles and long-distance vehicles in their respective time intervals. Finally, we can get the variance of hourly income of all taxis by considering the proportion of long-distance vehicles and short-distance vehicles. If the variance is reduced than before, it is proved that the scheme is feasible and effective.